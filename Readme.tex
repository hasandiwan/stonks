\documentclass{article}
\usepackage{listings}
\usepackage{noto}
\usepackage[colorlinks=true]{hyperref}
\begin{document}
\title{Stonks Bank Account}
\author{Hasan Diwan}
{
    \let\clearpage\relax
    \maketitle
}
\begin{center}
\maketitle
\end{center}

I'll save you the trouble... I have zero experience in the fintech
world, so don't look for it. However, what I do have is multiple
decades of detail-oriented, borderline obsessive commitment to make
the code I write well-documented such that it is painfully obvious at
a first glance that I don't know what I don't know. In that vein\ldots

You had asked how I'd deploy this package on AWS. The most
straightforward way to do so would be to use Lambda following
\href{https://docs.aws.amazon.com/lambda/latest/dg/python-package-create.html}{these}
directions.

Per bottlenecks in my design, the first issue that comes up has
already been raised in email -- the global nature of the account
limitations. My design has foregone account-specific withdrawal,
transaction, and top up limits for the reason that I'm keeping this
under 5 hours from start-to-finish (including documentation, running
tests, deploying, checking in, and composing the email to you guys
pointing you to said repository).

In terms of load, there's the well-known cPython global interpreter lock, which renders threading
practically useless in python. There are well-documented ways to get
around this -- Python 3.9's \href{https://docs.python.org/3/library/multiprocessing.html}{multiprocessing
  module} seeming to be the recommendation of choice right now. Others include
running the program under \href{https://ironpython.net/}{ironpython}
or \href{https://jython.org}{jython}, neither of which were attempted for reasons of time.

The log is a simple dictionary. This will also not be persistent, so will need to be shoved somewhere. I'd suggest leveraging syslog, if possible. 
\end{document}
